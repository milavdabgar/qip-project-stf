% abstract.tex (Abstract)

\chapter*{Abstract}

Cybersecurity threats continue to evolve rapidly, with malware infections causing significant damages to organizations worldwide. This project presents a comprehensive comparative study of machine learning and deep learning approaches for malware threat prediction, implemented as the System Threat Forecaster.

The research evaluates thirteen models: seven ML algorithms (Decision Tree, Random Forest, LightGBM, Naive Bayes, Logistic Regression, AdaBoost, SGD) and six DL architectures (Simple MLP, Deep MLP, Residual Network, Attention Network, Wide \& Deep, FT-Transformer). Using a dataset of 100,000 samples with 75 features (47 numerical, 28 categorical), we implemented a complete pipeline encompassing preprocessing, feature engineering, model training, and evaluation.

LightGBM achieved the best performance at 62.94\% validation accuracy (F1-score: 0.6286), outperforming all deep learning models. The best DL model (Deep MLP) reached 61.79\% accuracy with 63,714 parameters. This 1.15\% performance gap demonstrates that tree-based ensemble methods maintain superiority over neural networks for tabular data. All DL architectures clustered around 61.5\%, confirming that model complexity does not overcome dataset limitations.

Analysis revealed weak feature correlations (max 0.118), establishing a performance ceiling around 63\%. The top Kaggle score of 69.6\% represents only 6.7\% improvement over our baseline. This work includes a production deployment at \url{https://stf.milav.in} and provides evidence-based guidance for choosing ML over DL for tabular cybersecurity data.
