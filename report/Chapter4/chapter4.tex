\chapter{Results and Discussion}

Add contents-Description
Example: The experiments were conducted using Python 3.11.2 on an NVIDIA RTX A6000 workstation with 48GB RAM.

\section{Result}
Add contents-figures (Blocks output, validation accuracy plots, validation loss, confusion matrix)  & tables, & Performance parameters


\section{Discussion}
Add a contents:  table for comparison between the existing system and the proposed system, and describe it.
Example-Table~\ref {tab:performance_comparison} summarises the performance of the four models on the MSU dataset compared with existing literature.

\begin{table}[ht]
    \centering
    \caption{Comparative Performance of Developed Models and Literature}
    \begin{tabular}{|l|l|c|c|c|}
    \hline
    \textbf{Source} & \textbf{Model / Method} & \textbf{Accuracy} & \textbf{Sensitivity} & \textbf{Specificity} \\ \hline
    \multicolumn{5}{|c|}{\textbf{Existing Literature (MSU Dataset)}} \\ \hline
    \cite{ieee2025signaltoimage} & Signal-to-Image + CNN & 97.70\% & -- & -- \\ \hline
    \cite{banglajol2025dualtree} & Dual Tree CWT + SVM & $\approx$95.00\% & -- & -- \\ \hline
    \cite{mdpi2025diagnosis} & Markov Transition Fields & 98.51\% & 100.00\% & -- \\ \hline
    \multicolumn{5}{|c|}{\textbf{Proposed Models (This Work)}} \\ \hline
    Model 1 & Original Custom CNN & 82.57\% & 76.32\% & 89.72\% \\ \hline
    Model 2 & Enhanced Custom CNN & 81.87\% & 78.73\% & 85.46\% \\ \hline
    Model 3 & SE-CNN (Composite) & 85.85\% & 85.96\% & 85.71\% \\ \hline
    \textbf{Model 4} & \textbf{SE-CNN 256 (Independent)} & \textbf{93.43\%} & \textbf{96.00\%} & \textbf{89.99\%} \\ \hline
    \end{tabular}
    \label{tab:performance_comparison}
    \small{\textit{Note: Literature results may use record-wise validation. This work strictly uses LOSO for subject-wise evaluation.}}
\end{table}
