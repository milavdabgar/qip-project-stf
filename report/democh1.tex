\chapter{Latex Examples}

\section{Simple Text}          % This command makes a section title.

Words are separated by one or more spaces.  Paragraphs are separated by
one or more blank lines.  The output is not affected by adding extra
spaces or extra blank lines to the input file.

Double quotes are typed like this: ``quoted text''.
Single quotes are typed like this: `single-quoted text'.

Long dashes are typed as three dash characters---like this.

Emphasized text is typed like this: \emph{this is emphasized}.
Bold       text is typed like this: \textbf{this is bold}.

\subsection{A Warning or Two}  % This command makes a subsection title.

If you get too much space after a mid-sentence period---abbreviations
like etc.\ are the common culprits)---then type a backslash followed by
a space after the period, as in this sentence.

Remember, don't type the 10 special characters (such as dollar sign and
backslash) except as directed!  The following seven are printed by
typing a backslash in front of them:  \$  \&  \#  \%  \_  \{  and  \}.  
The manual tells how to make other symbols.

\section{Figures}

include figure in the report.
\begin{figure}[h]\label{fig1}
\includegraphics[scale=0.5]{ltex.png}
\centering
\caption[Latex logo]{Latex logo\cite{url1}}
\end{figure}

\begin{figure}[h]
\centering
\subfigure[First.]{\includegraphics[scale=0.5]{ltex}}\qquad
\subfigure[Second figure.]{\includegraphics[scale=0.5]{ltex}}\\
\subfigure[Third.]{\includegraphics[scale=0.5]{ltex}}%
\caption{Three subfigures.}
\label{3figs}
\end{figure}



\begin{figure}[h]
\centering
\begin{minipage}{0.5\textwidth}
  \centering
  \includegraphics[width=0.4\linewidth]{ltex}
  \captionof{figure}{A figure}
  \label{fig:test1}
\end{minipage}%
\begin{minipage}{0.5\textwidth}
  \centering
  \includegraphics[width=0.4\linewidth]{ltex}
  \captionof{figure}{Another figure}
  \label{fig:test2}
\end{minipage}
\caption{side by side figure}
\end{figure}

\newpage

\section{Table}
Example 1.
\begin{table}[h]
\centering
\caption{Number of Adders and Shifters required for 4-Point DCT.}
\resizebox{4.5in}{!}{
\begin{tabular}{|c|c|c|c|c|}
\hline
\textbf{} & \textbf{\begin{tabular}[c]{@{}c@{}}Proposed 1\\ 4-Point DCT\end{tabular}} & \textbf{\begin{tabular}[c]{@{}c@{}}Proposed 2\\ 4-Point DCT\end{tabular}} & \textbf{\begin{tabular}[c]{@{}c@{}}Meher \\ \textit{et al.} in\cite{PK}\end{tabular}} & \textbf{\begin{tabular}[c]{@{}c@{}}Optimization \\ of Meher \textit{et al.} in\cite{PK}\end{tabular}} \\ \hline
\textbf{\begin{tabular}[c]{@{}c@{}}No. Of\\ Adders\end{tabular}} & 16 & 12 & 14 & 11 \\ \hline
\textbf{\begin{tabular}[c]{@{}c@{}}No. Of\\ Shifters\end{tabular}} & 10 & 5 & 10 & 5 \\ \hline
\end{tabular}}
\end{table}

\end{document}                 % The input file ends with this command.
