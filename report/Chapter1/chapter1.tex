% chapter1.tex 

\chapter{Introduction}

\section{Background}

Malware has evolved into complex threats with organizations facing over 200,000 samples daily. Traditional signature-based detection struggles against zero-day attacks. Machine learning and deep learning enable proactive behavioral analysis. This project compares traditional ML with modern DL architectures using Kaggle competition data.

\section{Problem Statement}

Developing accurate malware prediction faces challenges: high dimensionality (75 features), weak correlations (max 0.118), missing data (1-30\%), mixed types (47 numerical, 28 categorical), and performance ceiling (~63\%).

\textbf{Research Questions}: How do ML algorithms compare with DL for tabular cybersecurity data? Can deep learning provide advantages over gradient boosting? What are practical complexity-performance trade-offs?

\section{Objectives and Contributions}

Implement robust preprocessing, train 13 models (7 ML + 6 DL), conduct systematic comparison, and deploy at \url{https://stf.milav.in}. Empirically validate that LightGBM (62.94\%) outperforms DL (61.79\%). Demonstrate that architectural complexity doesn't improve tabular performance.